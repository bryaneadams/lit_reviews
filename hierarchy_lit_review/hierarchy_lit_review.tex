\documentclass[12pt]{article}
\usepackage[top=1in, bottom=1in, left=1in, right=1in]{geometry}
\usepackage{amsmath}
\usepackage{algpseudocode}
\usepackage{amssymb}
\usepackage[T1]{fontenc}
\usepackage{hyperref}
\usepackage{graphicx}
\usepackage{makecell}
\usepackage{algorithm}
\usepackage{subcaption}
\usepackage{titlesec}

\pdfpagewidth 8.5in
\pdfpageheight 11.0in
\textheight = 700pt

\setlength{\parindent}{0pt}



\begin{document}

\subsection*{Scaling nonhierarchically: A theory of conflict-free organizational growth with limited hierarchical growth\cite{nonhierarchically}}

\subsubsection*{Abstract}
Research Summary We propose a theory that explains variations in the relationship between an organization's size and the extent of its authority hierarchy (as captured in managerial intensity). Conceptualizing authority hierarchy as a means to manage conflicts among subordinates, we formulate a model in which the number of managers required depends on the magnitude of conflicts generated between and within groups of workers. Our analysis shows that scaling non-hierarchically can be accomplished either by creating low conflict ``self-managing'' teams or reducing conflicts between many ``self-contained'' teams, but which path is more effective varies by situation. Small initial differences in terms of their emphasis on within vs. between team conflict mitigation can lead to large differences as firms scale over time in the extent of their authority hierarchies. Managerial Summary Managing without an extensive hierarchy can be attractive for a variety of reasons, but under what conditions is it possible in large scale organizations? We build on the premise that the managerial hierarchy of authority serves to resolve conflicts that employees cannot resolve peer-to-peer (i.e., there are limits to scaling groups that manage themselves consensually). We develop a formal theory that predicts that there are three levers that can slow down the growth of managerial hierarchy even as the organization scales: investing in the technology and culture needed to (a) expand managerial capacity particularly toward the apex of the hierarchy (b) create ``self-managed'' teams that produce few conflicts in need of managerial resolution and (c) create ``self-contained'' teams that generate few conflicts between them that need escalation up the hierarchy for resolution. The third is likely to be the most effective lever as organizations grow.

\subsubsection*{Summary}

Objective of the paper:  In this paper, we develop a theoretical framework to investigate the conditions under which an organization can scale (i.e., grow in terms of the number of its production workers) without significantly increasing the extent of its authority hierarchy.

there are three basic levers for ways to control scaling: first, increasing managerial capacity at higher layers; second, reducing the likelihood of conflicts requiring managerial intervention within teams (creating self-managed teams); and third, reducing the likelihood of conflicts requiring managerial intervention between teams (creating self-contained teams).

why larger less hierarchical organizations tend to emphasize self-contained teams more than smaller such organizations do (Laloux, 2014; Puranam \& Håkonsson, 2015).

As a result, if managerial capacity (and therefore the feasible span of managerial control) increases, then for a fixed number of employees, we should expect a reduction in the layers in the organization

The analysis of the three-layered hierarchy shows the intuitions behind how the shape of hierarchy as well as the scale of the organization at the end of the growth process depend on designs for self-managed and self-contained teams, as well as managerial capacity.


\subsubsection*{Data}

They do not have data but instead simulate based on inputs they have

\subsubsection*{Methods}

They simulate and control the input variables to determine optimal hierarchy structures based on their inputs. They consider a three-layered hierarchy which has a “layer 1” of production workers grouped in teams, a “layer 2” of middle managers (one for each team), and a “layer 3” CEO.

\subsubsection*{Application to my research}

We have team hierarchies and can measure the number of layers, span of control, ext. We could use this for an empirical study of managerial capacity.

We could look at coordinated moves as a possible signal of conflict or work capacity increase and how that changes the hierarchy in the future.

%%%%%-----------%%%%%

\subsection*{Hierarchies and the Organization of Knowledge in Production\cite{hierarchies_and_the_organization}}

\subsubsection*{This paper studies how communication allows for the specialized acquisition of knowledge. It shows that a knowledge-based hierarchy is a natural way to organize the acquisition of knowledge when matching problems with those who know how to solve them is costly. In such an organization, production workers acquire knowledge about the most common or easiest problems confronted, and specialized problem solvers deal with the more exceptional or harder problems. The paper shows that the model is consistent with stylized facts in the theory of organizations and uses it to analyze the impact of changes in production and information technology on organizational design.}

\subsubsection*{Summary}

Will cheaper communication technology make an organization taller or shorter? Looks at the tradeoff between knowledge and communication cost on production. Derives several formulas on propositions related to the size of hierarchies and the communication and sharing of knowledge (note not KSAs)

\subsubsection*{Data}

No data

\subsubsection*{Methods}

Purely mathematical

\subsubsection*{Application to my research}

Useful notation and formulas for annotating the sets of KSAs and potential use for the cost of transferring KSAs between hierarchies, teams and individuals. 

Proposition 4. Pyramidal organization.—An organization with multiple layers has a pyramidal structure, with each layer a smaller size than the previous one. - \textbf{could we find evidence of these hierarchy shapes}


\subsection*{The Dual Challenge of Search and Coordination for Organizational Adaptation: How Structures of Influence Matter\cite{dual_challenge}}

\subsubsection*{Abstract}
Organizations increasingly need to adapt to challenges in which search and coordination cannot be decoupled. In response, many have experimented with “agile” and “flat” designs that dismantle traditional forms of hierarchy to harness the distributed knowledge of specialized individuals. Despite the popularity of such practices, there is considerable variation in their implementation as well as conceptual ambiguity about the underlying prem- ise. Does effective rapid experimentation necessarily imply the repudiation of hierarchical structures of influence? We use computational models of multiagent reinforcement learning to study the effectiveness of coordinated search in groups that vary in how they influence each other’s beliefs. We compare the behavior of flat and hierarchical teams with a baseline structure without any influence on beliefs (a “crowd”) when all three are placed in the same task environments. We find that influence on beliefs—whether it is hierarchical or not— makes it less likely that agents stabilize prematurely around their own experiences. How- ever, flat teams can engage in excessive exploration, finding it difficult to converge on good alternatives, whereas hierarchical influence on beliefs reduces simultaneous uncoordinated exploration, introducing a degree of rapid exploitation. As a result, teams that need to achieve agility (i.e., rapid satisfactory results) in environments that require coordinated search may benefit from a hierarchical structure of influence—even when the apex actor has no superior knowledge, foresight, or capacity to control subordinates’ actions.

\subsubsection*{Summary}

\begin{enumerate}
    \item Miller et al. (2006) find that the level of organizational knowledge attained is superior when the influence structure is equivalent to a network with moderate degrees of clustering.
    \item Influence structures in organizations often have hierarchical patterns—denoting asymmetric, transitive, acyclic relationships (Simon 1962, Ahl and Allen 1996)
    \item Admittedly, influence is but one aspect of an administrative hierarchy, although arguably a crucial one.
    \item It is intuitive that if the apex actors in a hierarchical structure of influence are particularly knowledgeable, their downward influence can benefit the organization
    \item hierarchical structures of influence may have advantages either when coordination (not search) is the key imperative or in search situations when the more influential agents have superior knowledge or capabilities. 
    \item 
\end{enumerate}

\subsubsection*{Data}

\subsubsection*{Methods}

\subsubsection*{Application to my research}

Used an ABM to help show hierarchical structures help with knowledge sharing.

%%%%%-----------%%%%%

\bibliographystyle{plain}
\bibliography{refs}

\end{document}

%%%%%-----------%%%%%

% \subsection*{article}

% \subsubsection*{Abstract}

% \subsubsection*{Summary}

% \subsubsection*{Data}

% \subsubsection*{Methods}

% \subsubsection*{Application to my research}

%11. B. F. Ingram, G. R. Neumann, The returns to skill. Labour Econ. 13, 35–59 (2006).

