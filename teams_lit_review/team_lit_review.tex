\documentclass[12pt]{article}
\usepackage[top=1in, bottom=1in, left=1in, right=1in]{geometry}
\usepackage{amsmath}
\usepackage{algpseudocode}
\usepackage{amssymb}
\usepackage[T1]{fontenc}
\usepackage{hyperref}
\usepackage{graphicx}
\usepackage{makecell}
\usepackage{algorithm}
\usepackage{subcaption}
\usepackage{titlesec}

\pdfpagewidth 8.5in
\pdfpageheight 11.0in
\textheight = 700pt

\setlength{\parindent}{0pt}



\begin{document}

\subsection*{The Team Causes and Consequences of Team Membership Change: A Temporal Perspective\cite{team_causes}}

\subsubsection*{Abstract}

Membership change—adding, replacing, and losing members—is a common phenomenon in work teams and charts a different theoretical space from prior team research that has assumed stable team membership and shared team properties. Based on a comprehensive review of 133 empirical studies on team membership change since 1948, we pro- pose a temporal framework pertaining to the causes and consequences of membership change. Three key theoretical insights emerge from our evidence-based integration: (a) Membership change first disrupts team cognitive, behavioral, and interpersonal pro- cesses and states (e.g., transactive memory systems, coordination) but can benefit team performance after teams adapt to form new processes and states; (b) whether and to what extent team performance benefits from membership change is contingent on the magnitude of membership change, requirements of team communication, member adaptation-related attributes, change in team knowledge, skills, abilities, and other characteristics (KSAOs), and team knowledge work; and (c) poor team experiences motivate member departure and may make it challenging for newcomers to join and teams to adapt to membership change. Our review moves team research into new avenues that do not presume stable team membership and shared team properties in understanding team functioning and performance, and outlines key directions to advance integrative theory.


\subsubsection*{Summary}

This article discusses causes of change but focuses on their disruption and how long it takes for the teams to become productive again. \\

Some key findings that could be related to the research: \\

Integrative Conclusions about the Causes of Team Membership Change - When members have unpleasant experiences they would like to leave the team, when members have pleasant experiences they would like to stay. \\

What remains unclear in all this is the role of “team agency.”  would be highly worthwhile for future research to more systematically explore the active role teams may play in seeking to remove, add, or replace members, unexplored team decisions that concern potential value in team composition change (e.g., with a focus on member KSAOs) \\

Team dynamics that drive value-driven decisions and actions about membership change remain unexplored. \\

It discusses magnitude of change - proportion and if they are key or peripheral members. When discussing it talks about disruption not necessarily identity change. This means that members are more likely to leave the team, that more members are likely to leave the team (i.e., magnitude of membership change) \\

Team KASO change - The findings that membership change increasing KSAO level has less negative or more pos- itive effects directly follow from the former. Greater increases in team KSAOs may require greater magni- tude of membership change, for example


\subsubsection*{Data}

No data, this is an extensive literature review

\subsubsection*{Methods}

Lit review

%%%%%-----------%%%%%

\subsection*{From being diverse to becoming diverse: A dynamic team diversity theory\cite{diverse_to_becoming_diverse}}

\subsubsection*{Abstract}

On the basis of the literature of open systems and team diversity, we present a new dynamic team diversity theory that explains the effect of change in team diversity on team functioning and performance in the context of dynamic team composition. Building upon the conceptualization of teams as open systems, we describe the enlargement and decline of team variety, separation, and disparity through member addition, subtraction, and substitution. Then, focusing on diversity enlargement, we theorize the contemporaneous and lasting effects of team diversity change on team performance change and on team processes and states leading to them. Dynamic team diversity theory expands the focus of team diversity research from teams' being more diverse than others to teams' becoming more diverse than before. It aims to advance team diversity research to be better aligned with the organizational reality of dynamic team composition. We also discuss methodological considerations in subsequent empirical testing of the theory and highlight how the theory and future research may help to guide organizational practice in recomposing work teams.

\subsubsection*{Summary}

This paper proposes a new theory: theory suggests that team diversity in and of itself is a dynamic phenomenon and can enlarge or decline as teams add, subtract, and substitute members. \textbf{What they do argue is that any membership change implies some change in skills, knowledge, and abilities}. \\

With a dynamic team composition, team variety, separation, and disparity can enlarge or decline after team members are added, substituted, and subtracted in the team. Breaks the team changes into addition - new skills, substitution - replacing a skill, or subtraction - skill leaves. \\

It is likely that in the context of member substitution, team variety enlargement primarily disturbs established cognitive team states - what level of change disrupts a team to where they become "new" \\

They make propositions: \\

Proposition 1a. Team variety enlargement is negatively related to similarity of new TMMs developed in the same episode as variety enlargement (contemporaneous impact): The more team variety enlarges, the less similar new TMMs will be. \\

Proposition 1b. Similarity of new TMMs mediates the negative impact of team variety enlargement on change in team performance in the same episode (contemporaneous impact): The more team variety enlarges, the less similar new TMMs are, and hence, the more team performance will decline. \\

Proposition 1c. Team variety enlargement is positively related to subsequent changes in team performance in the following episodes after variety enlargement (lasting impact): The more team variety enlarges, the more team performance will improve in the new equilibrium.


\subsubsection*{Data}

None

\subsubsection*{Methods}

literature review

\subsubsection*{Application to my research}

This is a lit review, but it discusses \textbf{DYNAMIC TEAM DIVERSITY THEORY} which could be used for background information.\\\\

majority of research has treated teams as static entities with unchanging composition, whereas in organizational practice, the boundaries of work teams are in fact fluid\\\\

To date, research exploring team diversity in the temporal context has been conducted within the boundary of stable team composition.\\\\

DTDT sheds light on the implicit temporal boundaries of the three major theories of static team diversity (i.e., information processing, social categorization, and team hierarchy theories; Harrison \& Klein, 2007; Magee \& Galinsky, 2008; van Knippenberg \& Schippers, 2007).

%%%%%-----------%%%%%

\subsection*{Beyond Team Types and Taxonomies: A Dimensional Scaling Conceptualization for Team Description\cite{team_description}}

\subsubsection*{Abstract}

Research on teams has prompted the development of many alternative taxonomies but little consensus on how to differentiate team types. We show that there is greater consensus on the underlying dimensions differentiating teams than there is on how to use those dimensions to generate categorical team types. We leverage this literature to create a conceptual framework for differentiating teams that relies on a dimensional scaling approach with three underlying constructs: skill differentiation, author- ity differentiation, and temporal stability.

\subsubsection*{Summary}

They want to help create a framework to answer “What kind of team is this?”

The purpose of this article is to develop a new conceptual system for describing and differentiating different teams when building and testing theories. The three categories skill differentiation, authority differentiation, and tempo- ral stability constitute a parsimonious yet comprehensive and continuous three-dimensional conceptual space.\\

They discuss three taxonomies:  (1) skill differentiation — the degree to which members have specialized knowledge or functional capacities that make it more or less difficult to substitute members; (2) authority differentiation - the degree to which decision-making responsibility is vested in individual members, subgroups of the team, or the collective as a whole; and (3) temporal stability - the degree to which team members have a history of working together in the past and an expectation of work- ing together in the future.\\

Much of the empirical research on multiteam systems uses a small number of component teams (e.g., two), each made up of a small number of team members (two or three)\\

LePine et al. (2008) differentiated small teams from large teams, where they used the raw number of core team members as the measure of team size.\cite{meta_teamwork_processes} Because team size is a continuous variable, the use of the raw number of team members would seem a natural practice for operationalizing this variable. Still, the lure of categorical systems within the literature on teams is strong, and other researchers have categorized teams into small and large categories based on cutoff scores, where two or less reflected small and five or more was large (Salas et al., 2008; see Table 1, types 31 and 32). Salas states $n=2$ is small, $2 < n< 5$ as medium, and $n\geq 5$ is large.\cite{team_training}\\

Some temporal team relationships\\

Hackman (2002) refers to as real teams, where members may work together for as long as ten years. \\

high but slightly lower on this continuum are ongoing teams (Devine et al., 1999; see 14 and 16 in Table 1), intact teams (Salas et al., 2008; see 18 in Table 1), and long-term teams \\

useful definition - Management teams coordinate and provide direction to the sub-units under their jurisdiction, laterally integrating interdependent sub-units across key business processes ... The management team is responsible for the overall performance of a business unit. Its authority stems from the hierarchical rank of its members. It is composed of the managers responsible for each subunit” (Cohen \& Bailey, 1997: 243)


\subsubsection*{Data}

None

\subsubsection*{Methods}

literature review

%%%%%-----------%%%%%

\subsection*{A Review and Integration of Team Composition Models: Moving Toward a Dynamic and Temporal Framework\cite{dynamic_teams_review}}

\subsubsection*{Abstract}
Although teams are considered to be the building blocks of modern organizational designs and numerous theoretical models, and narrative and meta-analytic reviews of the literature exist, there is a lack of coherence, integration, and understanding of how team composition effects relate to important team outcomes. Accordingly we have five primary goals for this article. First, we categorize team composition models into four types and highlight theory and research associated with each one. Second, we offer an integrative framework that represents members’ attributes as simultaneously contributing variance to each of the four model types. Third, we overlay temporal considerations that suggest different team compositional mixes will be more or less salient at different periods of performance episodes or stages of team development. Fourth, we integrate membership dynamics into our model. And fifth, we advance an integrative optimization algorithm that incorporates implications from all for the four previous approaches, as well as temporal dynamics and membership change. In so doing, we provide a synthesis of previous work and theories and outline a research agenda for both research and practice. 

\subsubsection*{Summary}
\begin{enumerate}
    \item teams are considered to be the building blocks of modern organizational designs
    \item Gully (2000: 35) submitted that “to fully understand work teams, researchers must investigate how team dynamics develop and change over time.
    \item Are there important team requisite KSAs that are not embedded in any particular role or position?
    \item Do teams converge, diverge, or remain parallel in level of performance across time (i.e., what will happen to initial differences over time)?
    \item At what point does the process loss associated with member turnover get offset if the KSAs of the replacement are better than those of the leaver
    \item What (and how large) is the effect of previous work experience with team members in member replacement on team effectiveness?
\end{enumerate}


\subsubsection*{Data}

None

\subsubsection*{Methods}

literature review

\subsubsection*{Application to my research}

Had some good examples of empirical research.

%%%%%-----------%%%%%

\subsection*{Embracing Complexity: Reviewing the Past Decade of Team Effectiveness Research\cite{embracing_complexity}}

\subsubsection*{Abstract}
We conceptualize organizational teams as dynamic systems evolving in response to their environments. We then review the past 10 years of team effectiveness research and summarize its implications by categorizing studies under three main overlapping and coevolving dimensions: compositional features, structural features, and mediating mechanisms. We highlight prominent work that focused on variables in each of these dimensions and discuss their key relationships with team outcomes. Furthermore, we review how contextual factors impact team effectiveness. On the basis of this review, we advocate that future research seek to examine team relationships through a dynamic, multilevel perspective, while incorporating new and novel measurement techniques. We submit that the future of teams research may benefit from a conceptualization of them as dynamic networks and modeling them as small complex systems.
\subsubsection*{Summary}

\begin{enumerate}
    \item Wang et al. (2014) found that task complexity significantly strengthened the relationship between shared leadership and team performance, whereas D’Innocenzo et al. (2016) found the opposite,
    \item The number of members on any given team is an important factor in team effectiveness.
    \item As another exemplar, military action teams gained mental efficacy when they were larger in size, which positively influenced their effectiveness
    \item Hale et al. (2016) theorized how bank teams would adapt to turnover events. Utilizing a discontinuous growth model, the authors found that teams adapted similarly with the loss of either an employee or a manager but did not adapt as well with both an employee and a manager exit. Adaptation was even more difficult for highly interdependent teams. This study highlights the im- portance of understanding compositional and structural features of the team in its environment, and how these features are dynamic across time, in studying team adaptation. Ben-Menahem et al. (2016) provide a complementary methodological approach with their in-depth qualitative interviews of team adaptability in the pharmaceutical industry. Their findings showed that teams have both formal and informal processes for revealing their interdependencies and as these inter- dependencies are revealed, team adaptation to the new information occurs.
\end{enumerate}

\subsubsection*{Data}

None

\subsubsection*{Methods}

literature review

\subsubsection*{Application to my research}

Some quotes that are beneficial to and support the reason we need to understand team composition and changes over time.

%%%%%-----------%%%%%

\subsection*{Advancing research on team process dynamics: Theoretical, methodological, and measurement considerations\cite{advancing_research}}

\subsubsection*{Abstract}
Team processes are inherently dynamic phenomena theoretically, but they have largely been treated as static in research. It is well established that they are important contributors to team effectiveness; the lack of attention to dynamics means that team process mechanisms are essentially unstudied. I examine three primary themes. First, I speculate as to why it is that research treatments of team processes are largely static and what may account for this inertia. Second, I consider the conceptual underpinnings of process dynamics with respect to (a) emergence across levels and (b) in terms of variability, trajectory, and cyclical fluctuation. Third, I discuss three ways that research on team process dynamics can be advanced by: (a) adapting existing research methods, (b) creating innovative measurement techniques, and (c) advancing new research paradigms. Empirical examination of team dynamics is on the research frontier. These suggestions hold promise for advancing understanding of dynamic process mechanisms.

\subsubsection*{Summary}

"The more fundamental theory evaluation and research issues center on conceptualizing process dynamics and collecting data that are aligned with them. That is the focus of this paper."

\begin{enumerate}
    \item  I discuss three ways that research on team process dynamics can be advanced by: (a) adapting existing research methods, (b) creating innovative measurement techniques, and (c) advancing new research paradigms. Empirical examination of team dynamics is on the research frontier
    \item McGrath consistently championed the need for team research to incorporate process dynamics
    \item Empirical research tends to treat these different emergent forms as static and invariant.
    \item For example, team members working together collaboratively may develop a homogeneous knowledge representation of the task domain;
    \item Through knowledge building, team members acquire distinctive knowledge about a problem space that yields a configuration at the team level (compilation). As members exchange their unique knowledge, it becomes a shared property of the team (composition) that can be applied to problem solving (Kozlowski \& Chao, 2012a, 2012b; Kozlowski et al., 2013)
    \item discuss three potential types of process dynamics for a given phenomenon that focus on (a) within- team variability, (b) growth trajectories, and (c) fluctuations over time. 
    \begin{enumerate}
        \item Within-team variability is related to the stability and/or form of emergence for a particular phenomenon. 
        \item Another dynamic pattern would be indicated by growth trajectories that capture a pattern of linear increase or decrease in the level or amount of an emerged property over time.
        \item  there may be rhythmic or entrained cycles in process dynamics or, relatedly, reciprocal relationships.
    \end{enumerate}
    \item Questionnaires take time to complete, are obtrusive, entail a variety of response biases, are potentially distorted by more frequent assessments, and are not necessarily accurate.
    \item inherently limited and at least need to be supplemented by other measurement methods that assess behavior in real time, are unobtrusive, are reliable, and can be captured at higher sampling rates.
    \item Unfortunately, theories of team effectiveness and functioning lack precision with respect to temporal scaling and rates of process change.
\end{enumerate}

\subsubsection*{Data}

None

\subsubsection*{Methods}

Does discuss methods that could be used. It discusses ABMs which might be worth a re-read.

\subsubsection*{Application to my research}

Supporting arguments and potential methodological suggestions that could be applicable, such as, Agent Based Modeling.

%%%%%-----------%%%%%

\subsection*{Ties, leaders, and time in teams: strong inference about network structure's effects on team viability and performance\cite{time_in_teams_strong_inference}}

\subsubsection*{Abstract}

How do members’ and leaders’ social network structures help or hinder team effectiveness? A meta-analysis of 37 studies of teams in natural contexts suggests that teams with densely configured interpersonal ties attain their goals better and are more committed to staying together; that is, team task performance and viability are both higher. Further, teams with leaders who are central in the teams’ intragroup networks and teams that are central in their intergroup network tend to perform better. Time sequencing, member familiarity, and tie content moderate structure-performance connections. Results suggest stronger incorporation of social network concepts into theories about team effectiveness.

\subsubsection*{Summary}
\begin{enumerate}
    \item The structure of a social network is the pattern of connections among parties; the parties are generically referred to as nodes (Nadel, 1957)
    \item Put simply, social ties in work teams are informal links between team members. Teams in which many members have ties to one another (i.e., high- density teams) should therefore have higher levels of information sharing and more of the collaboration necessary for successful task completion.
    \item teams with sparse networks might have to rely on individuals to act as brokers between disconnected parts of the team
\end{enumerate}

\subsubsection*{Data}

None

\subsubsection*{Methods}

literature review

\subsubsection*{Application to my research}

Had some good idea of background topics and some discussion forward that might be worthwhile formatting.

%%%%%-----------%%%%%

\subsection*{Becoming Team Players: Team Members’ Mastery of Teamwork Knowledge as a Predictor of Team Task Proficiency and Observed Teamwork Effectiveness\cite{becoming_team_player}}

\subsubsection*{Abstract}
The authors explored the idea that teams consisting of members who, on average, demonstrate greater mastery of relevant teamwork knowledge will demonstrate greater task proficiency and observed teamwork effectiveness. In particular, the authors posited that team members’ mastery of designated teamwork knowledge predicts better team task proficiency and higher observer ratings of effective teamwork, even while controlling for team task proficiency. The authors investigated these hypotheses by developing a structural model and testing it with field data from 92 teams (1,158 team members) in a United States Air Force officer development program focusing on a transportable set of teamwork competencies. The authors obtained proficiency scores on 3 different types of team tasks as well as ratings of effective teamwork from observers. The empirical model supported the authors’ hypotheses.

\subsubsection*{Summary}

Looked at teams in the Air Force and tasks they had to complete. 
\begin{enumerate}
    \item Schmidt and Hunter (2004) underscored the acquisition of job knowledge (i.e., learning) as the key explanatory factor through which individuals’ general mental ability translates into individual success in the world of work.
    \item 
\end{enumerate}

\subsubsection*{Data}

\begin{enumerate}
    \item 92 teams (1,158 team members)
    \item All of the teams in this study were newly formed (i.e., observed relationships were not influenced by team phenomena occurring prior to the ODP)
    \item The ODP was conducted at a large USAF base, and all 1,158 participants were full-time USAF officers with 5-7 years of commissioned service. Participants averaged 31 years of age; and 83\% of them were men, and 17\% were women.
    \item Each team member independently completed a multiple-choice test on Friday of Week 2. The test measured mastery of the teamwork concepts or principles that were presented or reviewed in hundreds of pages of readings and the roughly 40 hr of classroom activities during the first 2 weeks of the ODP
    \item To obtain observer ratings of teamwork, for research purposes only, we asked each team’s observer to assess his or her team’s behavior over the course of the ODP
    \item 
\end{enumerate}

\subsubsection*{Methods}

Used basic statistic techniques to compare the team performance across categories.

\subsubsection*{Application to my research}

Might help with KSAs

%%%%%-----------%%%%%

\subsection*{Time matters in team performance: Effects of member familiarity, entrainment and task discontinuity on speed and quality.\cite{time_matters}}

\subsubsection*{Abstract}
We compared the speed and quality of performance for familiar,initially unfamiliar but continuing, and one-shot (single session) teams. We also proposed and observed entrainment effects for task time lim- its. Over the course of weekly sessions with changing tasks, continuing teams reached speed levels of the initially familiar teams, but the one- shot teams were consistently slower. Continuing teams also tended to have higher-quality output than the one-shot teams. There were no differences in how quickly each type of group entrained to time limits on the tasks. Entrainment was not robust to task discontinuity (Xsk A, then B). However, entrainment on repeated trials of a task persisted even when a different type of task “interrupted” those repeated trials (Task A, then B, then A again). Results compel a richer incorporation of time as a medium for complex task sequences, and time-based constructs as a feature of team membership in the study of group effectiveness.

\subsubsection*{Summary}

\begin{enumerate}
    \item lore recently, Marks, Mathieu, and Zaccaro (2001) proposed a framework of team processes in which team performance is viewed as a series of related input-process-outcome episodes over time.
    \item McGrath’s (1984) time-based framework for differentiating teani types distinguishes three broad categories-natural, concocted, and quasi-groups, which are then further differentiated into 12 subtypes.
    \item One form of concocted group, the taskforce, can have a lifetime that varies from a few days to several years.
    \item 
\end{enumerate}

\subsubsection*{Data}
\begin{enumerate}
    \item longitudinal experiment over a series of three task sessions, each separated by one week. We collected data on speed and quality of performance during each session for each group type: familiar, continuing, and one-shot.
    \item participants were 18-25year-old students from a large mid- western university. The study began with 72 triads, 8 triads for each cell in the design (216 individual participants)
    \item  familiar teams with strong ties, we needed to choose individuals that had already been together for some time
    \item continuing and one-shot team conditions came from a pool of introductory psychology students
    \item Students in one-shot teams were randomly assigned in the same man- ner as continuing teams. The important temporal difference between continuing and one-shot teams was that latter met for only one session.
\end{enumerate}

\subsubsection*{Methods}

did conduct an empirical analysis, but methods are not necessarily applicable to my research. 

\subsubsection*{Application to my research}

Has some potential background


%%%%%-----------%%%%%

\subsection*{Talent and time together: The impact of human capital and overlapping tenure on unit performance\cite{talent_and_time}}

\subsubsection*{Abstract}
Purpose – This study aims to directly examine the relationships between various aspects of human capital and relationship stability (overlapping tenure) and team performance. Additionally, this study aims to contribute to strategic human resource management and human capital research by placing an emphasis on human resources (i.e. people) and their influence on performance. \\\\
Design/methodology/approach – The direct and interaction effects of human capital and overlapping tenure on performance are examined with a sample of 230 National Collegiate Athletic Association (NCAA) men’s basketball teams in the 2006-2007 season. A third party measure of basketball players’ human capital is aggregated to the team level to examine its relationship with team performance. Additionally, the human capital of the head coach of each team and its relationship with team performance is examined. Relationship stability is assessed by measuring overlapping tenure, which is defined as the amount of time individuals have worked together towards a common performance outcome. Team level overlapping tenure among players and the overlapping tenure between players and their head coach are measured and their relationships with team performance are tested. Finally, the interaction effect of players’ human capital and players’ overlapping tenure on team performance is examined. Hierarchical regression is used to test each hypothesis.\\\\
Findings – The results find a positive relationship between both players’ and coaches’ human capital and performance. Also, players’ overlapping tenure is positively related to performance. Lastly, the interaction between players’ human capital and players’ overlapping tenure is not significantly related to performance.\\\\
Originality/value – There has been a growing interest in human resources (i.e. people) as a source of competitive advantage. This study employs a unique sample of NCAA men’s basketball teams to theoretically develop and empirically test relationships among human capital, overlapping tenure, and performance. Different from previous studies, an objective, third party measure of human capital and measurements of overlapping tenure are utilized and their direct and interaction effects on team performance are examined. The results of this study point to the importance of acquiring and retaining high levels of human capital.

\subsubsection*{Summary}

This study followed the resource-based view, human capital, and relationship stability literature and empirically demonstrated the influence human resources can have on unit performance

\begin{enumerate}
    \item Thus, hiring and retaining human resources with high levels of human capital may be difficult and costly for an organization; however an organization that is able to accomplish this may achieve a competitive advantage.
    \item Research on the relationship between leader quality and performance has frequently examined coaches of athletic teams.
    \item Studies that have examined team tenure have tended to find positive outcomes associated with longer team tenure. For example, teams with higher levels of experience performed at a higher level (e.g., Berman et al., 2002; Timmerman, 2005).
    \item Additionally, shared team experience has been found to have a positive influence on performance (e.g. Berman et al., 2002; Pelled et al., 1999). The shared experiences of individuals may allow them to coordinate their activities better, share knowledge, learn, and therefore perform at higher levels (Berman et al., 2002).
    \item 
\end{enumerate}

\subsubsection*{Data}

\begin{enumerate}
    \item They looked at NCAA basketball teams:  Team size. The number of players on each team was also used as a control variable. The average team size was 14.3. The teams in this sample had between 10 and 19 members.
\end{enumerate}

\subsubsection*{Methods}

Used regression analysis

\subsubsection*{Application to my research}

Good background on why the time horizon is important, but again empirical analysis is not really linked to our research. 

%%%%%-----------%%%%%

\subsection*{Personality, Five factor model, Intelligence, Group influence, Group leadership\cite{personality_five_factor}}

\subsubsection*{Abstract}

The ability of personality and cognitive ability to predict perceptions of group influence in small work groups are assessed both in initial and advanced stages of group formation. Extraversion is found important to initial perceptions of intra-group influence, which is partially mediated by peer-perceived social-emotional usefulness. After a few months, reputations are established and everyone has met; now work needs to get done efficiently and accurately and cognitive ability predicts increases in perceived group influence, which is partially mediated by perceived intelligence. After even more time, other Big Five personality traits become important to changes in perceived group influence, with positive associations with openness to experience, and negative associations with neuroticism and conscientiousness. The study findings and implications are discussed.

\subsubsection*{Summary}

\begin{enumerate}
    \item Intelligence or cognitive ability is one of the most frequently studied individual differences, and is a robust predictor of many performance outcomes, such as job performance and skill acquisition in training
    \item most studies still use designs that empirically treat groups as a static entity. The present study goes beyond such an approach by considering changing performance requirements as driving the changes in antecedents of group influence over time, which is in accordance with the concept of dynamic criteria (see Steele-Johnson, Osburn, \& Pieper, 2000 for a review).
\end{enumerate}

\subsubsection*{Data}
\begin{enumerate}
    \item Participants were psychology freshmen attending Utrecht University in the Netherlands.
    \item randomly placed in groups in which they work together during the entire academic year to complete a substantial part of the psychology curriculum. The students were as- signed to one of the 20 introduction groups of around 25 people each (N = 489).
    \item 378 participants (77\% of all first-year stu- dents) stemming from 18 groups signed up for the study via a website
    \item Dropout amounted to 16 participants between Waves 1 and 2, and 10 participants be- tween Wave 2 and Wave 3 (88\% retention rate)
    \item Thus future research on the association between group influence, an important component of leadership, and individual differences should consider the moderating role of group age or group stage.
    \item Longitudinal studies such as the current one are important to specify under which circumstances and developmental phases a specific trait may affect a specific group outcome.
\end{enumerate}

\subsubsection*{Methods}

Used regression analysis to look at how teams change over time. If you know person A has KSA, what does the next team look like?\\\\



\subsubsection*{Application to my research}

Provides good background information on why longitudinal studies are important. Has some KSA relationships.

%%%%%-----------%%%%%

\subsection*{Data-driven modeling of collaboration networks: a cross-domain analysis\cite{data_driven_modeling}}

\subsubsection*{Abstract}

We analyze large-scale data sets about collaborations from two different domains: economics, specifically 22,000 R\&D alliances between 14,500 firms, and science, specifically 300,000 co-authorship relations between 95,000 scientists. Considering the different domains of the data sets, we address two questions: (a) to what extent do the collaboration networks reconstructed from the data share common structural features, and (b) can their structure be reproduced by the same agent-based model. In our data-driven modeling approach we use aggregated network data to calibrate the probabilities at which agents establish collaborations with either newcomers or established agents. The model is then validated by its ability to reproduce network features not used for calibration, including distributions of degrees, path lengths, local clustering coefficients and sizes of disconnected components. Emphasis is put on comparing domains, but also sub-domains (economic sectors, scientific specializations). Interpreting the link probabilities as strategies for link formation, we find that in R\&D collaborations newcomers prefer links with established agents, while in co-authorship relations newcomers prefer links with other newcomers. Our results shed new light on the long-standing question about the role of endogenous and exogenous factors (i.e., different information available to the initiator of a collaboration) in network formation.

\subsubsection*{Summary}

\begin{enumerate}
    \item large-scale and time resolved data sets about economic, scientific or social activities opens new venues to address the long standing question of how we collaborate.
    \item hypothesize that a unified modeling approach should be able to reproduce, and to explain, the structural and the dynamic features of collaborations in different domains
    \item reconstruction of the collaboration networks using the empirical data from two different domains
    \item agent-based model is correct if it is able to reproduce a specific set of macroscopic properties of the different collaboration networks, namely degree distribution, path length distribution, distribution of community sizes, that are not used for the calibration of the model
\end{enumerate}

\subsubsection*{Data}

\begin{enumerate}
    \item To reconstruct the collaboration network of scientists, we use the data set from the American Physical Society about papers published in any APS journal, namely Physical Review Letters, Reviews of Modern Physics, and all Physical Review journals
    \item The latter involves matching the titles of the papers in the APS data set with Microsoft Academic Search (MSAS) service, where both papers and authors have unique identifiers.
    \item It is worth noticing that the matching procedure at article level was not perfect. About 27\% of the articles were not matched
    \item 
\end{enumerate}

\subsubsection*{Methods}

They did use Infomap

\begin{enumerate}
    \item community partitions detected by Infomap, we use a normalized modularity score Q. This coefficient is equal to 1 when all links connect only nodes belonging to the same community, equal to 0 for a network where links are placed randomly, and equal to -1 when links are formed only among nodes populating distinct communities.
\end{enumerate}

\subsubsection*{Application to my research}

%%%%%-----------%%%%%

\bibliographystyle{plain}
\bibliography{refs}

\end{document}

%%%%%-----------%%%%%

% \subsection*{article}

% \subsubsection*{Abstract}

% \subsubsection*{Summary}

% \subsubsection*{Data}

% \subsubsection*{Methods}

% \subsubsection*{Application to my research}

%11. B. F. Ingram, G. R. Neumann, The returns to skill. Labour Econ. 13, 35–59 (2006).

