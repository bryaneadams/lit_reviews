\documentclass[12pt]{article}
\usepackage[top=1in, bottom=1in, left=1in, right=1in]{geometry}
\usepackage{amsmath}
\usepackage{algpseudocode}
\usepackage{amssymb}
\usepackage[T1]{fontenc}
\usepackage{hyperref}
\usepackage{graphicx}
\usepackage{makecell}
\usepackage{algorithm}
\usepackage{subcaption}
\usepackage{titlesec}

\pdfpagewidth 8.5in
\pdfpageheight 11.0in
\textheight = 700pt

\setlength{\parindent}{0pt}



\begin{document}

\subsection*{The Team Causes and Consequences of Team Membership Change: A Temporal Perspective\cite{team_causes}}

\subsubsection*{Abstract}

Membership change—adding, replacing, and losing members—is a common phenomenon in work teams and charts a different theoretical space from prior team research that has assumed stable team membership and shared team properties. Based on a comprehensive review of 133 empirical studies on team membership change since 1948, we pro- pose a temporal framework pertaining to the causes and consequences of membership change. Three key theoretical insights emerge from our evidence-based integration: (a) Membership change first disrupts team cognitive, behavioral, and interpersonal pro- cesses and states (e.g., transactive memory systems, coordination) but can benefit team performance after teams adapt to form new processes and states; (b) whether and to what extent team performance benefits from membership change is contingent on the magnitude of membership change, requirements of team communication, member adaptation-related attributes, change in team knowledge, skills, abilities, and other characteristics (KSAOs), and team knowledge work; and (c) poor team experiences motivate member departure and may make it challenging for newcomers to join and teams to adapt to membership change. Our review moves team research into new avenues that do not presume stable team membership and shared team properties in understanding team functioning and performance, and outlines key directions to advance integrative theory.


\subsubsection*{Summary}

This article discusses causes of change but focuses on their disruption and how long it takes for the teams to become productive again. \\

Some key findings that could be related to the research: \\

Integrative Conclusions about the Causes of Team Membership Change - When members have unpleasant experiences they would like to leave the team, when members have pleasant experiences they would like to stay. \\

What remains unclear in all this is the role of “team agency.”  would be highly worthwhile for future research to more systematically explore the active role teams may play in seeking to remove, add, or replace members, unexplored team decisions that concern potential value in team composition change (e.g., with a focus on member KSAOs) \\

Team dynamics that drive value-driven decisions and actions about membership change remain unexplored. \\

It discusses magnitude of change - proportion and if they are key or peripheral members. When discussing it talks about disruption not necessarily identity change. This means that members are more likely to leave the team, that more members are likely to leave the team (i.e., magnitude of membership change) \\

Team KASO change - The findings that membership change increasing KSAO level has less negative or more pos- itive effects directly follow from the former. Greater increases in team KSAOs may require greater magni- tude of membership change, for example


\subsubsection*{Data}

No data, this is an extensive literature review

\subsubsection*{Methods}

Lit review

\subsection*{From being diverse to becoming diverse: A dynamic team diversity theory\cite{diverse_to_becoming_diverse}}

\subsubsection*{Abstract}

On the basis of the literature of open systems and team diversity, we present a new dynamic team diversity theory that explains the effect of change in team diversity on team functioning and performance in the context of dynamic team composition. Building upon the conceptualization of teams as open systems, we describe the enlargement and decline of team variety, separation, and disparity through member addition, subtraction, and substitution. Then, focusing on diversity enlargement, we theorize the contemporaneous and lasting effects of team diversity change on team performance change and on team processes and states leading to them. Dynamic team diversity theory expands the focus of team diversity research from teams' being more diverse than others to teams' becoming more diverse than before. It aims to advance team diversity research to be better aligned with the organizational reality of dynamic team composition. We also discuss methodological considerations in subsequent empirical testing of the theory and highlight how the theory and future research may help to guide organizational practice in recomposing work teams.

\subsubsection*{Summary}

This paper proposes a new theory: theory suggests that team diversity in and of itself is a dynamic phenomenon and can enlarge or decline as teams add, subtract, and substitute members. \textbf{What they do argue is that any membership change implies some change in skills, knowledge, and abilities}. \\

With a dynamic team composition, team variety, separation, and disparity can enlarge or decline after team members are added, substituted, and subtracted in the team. Breaks the team changes into addition - new skills, substitution - replacing a skill, or subtraction - skill leaves. \\

It is likely that in the context of member substitution, team variety enlargement primarily disturbs established cognitive team states - what level of change disrupts a team to where they become "new" \\

They make propositions: \\

Proposition 1a. Team variety enlargement is negatively related to similarity of new TMMs developed in the same episode as variety enlargement (contemporaneous impact): The more team variety enlarges, the less similar new TMMs will be. \\

Proposition 1b. Similarity of new TMMs mediates the negative impact of team variety enlargement on change in team performance in the same episode (contemporaneous impact): The more team variety enlarges, the less similar new TMMs are, and hence, the more team performance will decline. \\

Proposition 1c. Team variety enlargement is positively related to subsequent changes in team performance in the following episodes after variety enlargement (lasting impact): The more team variety enlarges, the more team performance will improve in the new equilibrium.


\subsubsection*{Data}

None

\subsubsection*{Methods}

literature review

\subsection*{Beyond Team Types and Taxonomies: A Dimensional Scaling Conceptualization for Team Description\cite{team_description}}

\subsubsection*{Abstract}

Research on teams has prompted the development of many alternative taxonomies but little consensus on how to differentiate team types. We show that there is greater consensus on the underlying dimensions differentiating teams than there is on how to use those dimensions to generate categorical team types. We leverage this literature to create a conceptual framework for differentiating teams that relies on a dimensional scaling approach with three underlying constructs: skill differentiation, author- ity differentiation, and temporal stability.

\subsubsection*{Summary}

They want to help create a framework to answer “What kind of team is this?”

The purpose of this article is to develop a new conceptual system for describing and differentiating different teams when building and testing theories. The three categories skill differentiation, authority differentiation, and tempo- ral stability constitute a parsimonious yet comprehensive and continuous three-dimensional conceptual space.\\

They discuss three taxonomies:  (1) skill differentiation — the degree to which members have specialized knowledge or functional capacities that make it more or less difficult to substitute members; (2) authority differentiation - the degree to which decision-making responsibility is vested in individual members, subgroups of the team, or the collective as a whole; and (3) temporal stability - the degree to which team members have a history of working together in the past and an expectation of work- ing together in the future.\\

Much of the empirical research on multiteam systems uses a small number of component teams (e.g., two), each made up of a small number of team members (two or three)\\

LePine et al. (2008) differentiated small teams from large teams, where they used the raw number of core team members as the measure of team size.\cite{meta_teamwork_processes} Because team size is a continuous variable, the use of the raw number of team members would seem a natural practice for operationalizing this variable. Still, the lure of categorical systems within the literature on teams is strong, and other researchers have categorized teams into small and large categories based on cutoff scores, where two or less reflected small and five or more was large (Salas et al., 2008; see Table 1, types 31 and 32). Salas states $n=2$ is small, $2 < n< 5$ as medium, and $n\geq 5$ is large.\cite{team_training}\\

Some temporal team relationships\\

Hackman (2002) refers to as real teams, where members may work together for as long as ten years. \\

high but slightly lower on this continuum are ongoing teams (Devine et al., 1999; see 14 and 16 in Table 1), intact teams (Salas et al., 2008; see 18 in Table 1), and long-term teams \\

useful definition - Management teams coordinate and provide direction to the sub-units under their jurisdiction, laterally integrating interdependent sub-units across key business processes ... The management team is responsible for the overall performance of a business unit. Its authority stems from the hierarchical rank of its members. It is composed of the managers responsible for each subunit” (Cohen \& Bailey, 1997: 243)


\subsubsection*{Data}

None

\subsubsection*{Methods}

literature review

%%%%%-----------%%%%%

\bibliographystyle{plain}
\bibliography{refs}

\end{document}

%%%%%-----------%%%%%

% \subsection*{article}

% \subsubsection*{Abstract}

% \subsubsection*{Summary}

% \subsubsection*{Data}

% \subsubsection*{Methods}

%11. B. F. Ingram, G. R. Neumann, The returns to skill. Labour Econ. 13, 35–59 (2006).

