\documentclass[12pt]{article}
\usepackage[top=1in, bottom=1in, left=1in, right=1in]{geometry}
\usepackage{amsmath}
\usepackage{algpseudocode}
\usepackage{amssymb}
\usepackage[T1]{fontenc}
\usepackage{hyperref}
\usepackage{graphicx}
\usepackage{makecell}
\usepackage{algorithm}
\usepackage{subcaption}
\usepackage{titlesec}

\pdfpagewidth 8.5in
\pdfpageheight 11.0in
\textheight = 700pt

\setlength{\parindent}{0pt}

\begin{document}

\subsection*{Organizational Misfits and the Origins of Brokerage in Intrafirm Networks\cite{misfits}}

\subsubsection*{Abstract}
To extend research on the effects of networks for career outcomes, this paper examines how career processes shape network structure. I hypothesize that brokerage results from two distinct mechanisms: links with former coworkers and with friends of friends accumulated as careers unfold. Furthermore, I hypothesize that “organizational misfits”—people who followed career trajectories that are atypical in their organization—will have access to more valuable brokerage opportunities than those whose careers followed more conventional paths. I tested this hypothesis with career history data recorded longitudinally for 30,000 employees in a large information technology firm over six years and sequence-analyzed to measure individual-level fit with typical career paths in the organization. Network position was measured using a unique data set of over 250 million electronic mail messages. Empirical results support the hypotheses that diverse, and especially atypical, careers have an effect on brokerage through mechanisms rooted in social capital, even when accounting for endogeneity between networks and mobility. In theorizing about misfit from prototypical patterns, this paper offers a new, theory-driven application of sequence-analytic methods as well as a novel measure of brokerage based on interactions across observable boundaries, a complement to the structural constraint measure based on interactions across holes in social structure.

\subsubsection*{Summary}

A broker is one who connects people and groups that are otherwise disconnected in the informal network structure. 

hypotheses:

\begin{enumerate}
    \item Baseline hypothesis: A diverse intraorganizational career history increases one's brokerage across social and organizational boundaries
    \item Hypothesis 2: Sharing mutual acquaintances will increase the likelihood that two organizationally distant people will be linked by a bridging tie
    \item Hypothesis 3: An individual's deviation from prototypical career trajectories in the organization gives rise to brokerage, even when accounting for diversity of experience.
\end{enumerate}



\subsubsection*{Data}

The data is for one company, referenced as BigCo. They look at 66 people who communicated with an additional 30,262 U.S. employees during the three-month period of the e-mail data.

They also had the work histories that where associated with the individuals

\subsubsection*{Methods}

They looked at the probability a person from category A to email a person category B. The categories are based on business unit, function, office, salary band.

They then looked at if a person from Category A had a low probability of emailing a person from category B, but they do. This is a person that connects unlikely groups and most likely is a "misfit", that is a non-traditional member.

\subsubsection*{Application to my research}

\begin{enumerate}
    \item Could we ID a career trajectory that makes a person a broker or what impacts the trajectory, e.g. being part of a coordinated move.
    \item Identify how much a coordinated move (management decision) influences a career trajectory and being a misfit.
    \item Can an identify atypical career trajectories - by OCS, location, interactions?
    \item the paper states "human capital may also explain some of the effect of career diversity on brokerage" but they do not have the data, could I use job ads to help measure human capital.
\end{enumerate}

%%%%%-----------%%%%%

\bibliographystyle{plain}
\bibliography{refs}

\end{document}

%%%%%-----------%%%%%

% \subsection*{article}

% \subsubsection*{Abstract}

% \subsubsection*{Summary}

% \subsubsection*{Data}

% \subsubsection*{Methods}

% \subsubsection*{Application to my research}

%11. B. F. Ingram, G. R. Neumann, The returns to skill. Labour Econ. 13, 35–59 (2006).

